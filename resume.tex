%-------------------------
% Resume in Latex
% Author : Jake Gutierrez
% Based off of: https://github.com/sb2nov/resume
% License : MIT
%------------------------

\documentclass[letterpaper,11pt]{article}

\usepackage{latexsym}
\usepackage[empty]{fullpage}
\usepackage{titlesec}
\usepackage{marvosym}
\usepackage[usenames,dvipsnames]{color}
\usepackage{verbatim}
\usepackage{enumitem}
\usepackage[hidelinks]{hyperref}
\usepackage{fancyhdr}
\usepackage[english]{babel}
\usepackage{tabularx}
\input{glyphtounicode}


%----------FONT OPTIONS----------
% sans-serif
% \usepackage[sfdefault]{FiraSans}
% \usepackage[sfdefault]{roboto}
% \usepackage[sfdefault]{noto-sans}
% \usepackage[default]{sourcesanspro}

% serif
% \usepackage{CormorantGaramond}
% \usepackage{charter}


\pagestyle{fancy}
\fancyhf{} % clear all header and footer fields
\fancyfoot{}
\renewcommand{\headrulewidth}{0pt}
\renewcommand{\footrulewidth}{0pt}

% Adjust margins
\addtolength{\oddsidemargin}{-0.5in}
\addtolength{\evensidemargin}{-0.5in}
\addtolength{\textwidth}{1in}
\addtolength{\topmargin}{-.5in}
\addtolength{\textheight}{1.0in}

\urlstyle{same}

\raggedbottom
\raggedright
\setlength{\tabcolsep}{0in}

% Sections formatting
\titleformat{\section}{
  \vspace{-4pt}\scshape\raggedright\large
}{}{0em}{}[\color{black}\titlerule \vspace{-5pt}]

% Ensure that generate pdf is machine readable/ATS parsable
\pdfgentounicode=1

%-------------------------
% Custom commands
\newcommand{\resumeItem}[1]{
  \item\small{
    {#1 \vspace{-2pt}}
  }
}

\newcommand{\resumeSubheading}[4]{
  \vspace{-2pt}\item
    \begin{tabular*}{0.97\textwidth}[t]{l@{\extracolsep{\fill}}r}
      \textbf{#1} & #2 \\
      \textit{\small#3} & \textit{\small #4} \\
    \end{tabular*}\vspace{-7pt}
}

\newcommand{\resumeSubheadingExtraSubtitle}[6]{
  \vspace{-2pt}\item
    \begin{tabular*}{0.97\textwidth}[t]{l@{\extracolsep{\fill}}r}
      \textbf{#1} & #2 \\
      {\small#3} & \textit{\small #4} \\
      \textit{\small#5} & \textit{\small #6} \\
    \end{tabular*}\vspace{-7pt}
}

\newcommand{\resumeSubSubheading}[2]{
    \item
    \begin{tabular*}{0.97\textwidth}{l@{\extracolsep{\fill}}r}
      \textit{\small#1} & \textit{\small #2} \\
    \end{tabular*}\vspace{-7pt}
}

\newcommand{\resumeProjectHeading}[2]{
    \item
    \begin{tabular*}{0.97\textwidth}{l@{\extracolsep{\fill}}r}
      \small#1 & #2 \\
    \end{tabular*}\vspace{-7pt}
}

\newcommand{\resumeSubItem}[1]{\resumeItem{#1}\vspace{-4pt}}

\renewcommand\labelitemii{$\vcenter{\hbox{\tiny$\bullet$}}$}

\newcommand{\resumeSubHeadingListStart}{\begin{itemize}[leftmargin=0.15in, label={}]}
\newcommand{\resumeSubHeadingListEnd}{\end{itemize}}
\newcommand{\resumeItemListStart}{\begin{itemize}}
\newcommand{\resumeItemListEnd}{\end{itemize}\vspace{-5pt}}

%-------------------------------------------
%%%%%%  RESUME STARTS HERE  %%%%%%%%%%%%%%%%%%%%%%%%%%%%


\begin{document}

%----------HEADING----------
% \begin{tabular*}{\textwidth}{l@{\extracolsep{\fill}}r}
%   \textbf{\href{http://sourabhbajaj.com/}{\Large Sourabh Bajaj}} & Email : \href{mailto:sourabh@sourabhbajaj.com}{sourabh@sourabhbajaj.com}\\
%   \href{http://sourabhbajaj.com/}{http://www.sourabhbajaj.com} & Mobile : +1-123-456-7890 \\
% \end{tabular*}

\begin{center}
    \textbf{\Huge \scshape Michael Vanden Heuvel} \\ \vspace{1pt}
    \small (000) 000-0000 $|$ \href{}{\underline{me at michaelmvh.com}} $|$ 
    \href{https://www.linkedin.com/in/michaelmvh}{\underline{linkedin.com/in/michaelmvh}} $|$
    \href{https://www.github.com/michaelmvh}{\underline{github.com/michaelmvh}} $|$
    \href{https://www.michaelmvh.com/}{\underline{michaelmvh.com}}
\end{center}


%-----------EDUCATION-----------
\section{Education}
  \resumeSubHeadingListStart
    \resumeSubheading
      {University of Wisconsin - Madison}{Madison, WI}
      {Bachelor of Science in Computer Science, Certificate in Business Fundamentals}{Sep. 2018 -- Dec. 2022}
  \resumeSubHeadingListEnd


%----------- Industry EXPERIENCE-----------
\section{Industry Experience}
  \resumeSubHeadingListStart
    \resumeSubheading
      {Microsoft}{March 2023 -- Present}
      {Microsoft Sentinel - Software Engineer}{Redmond, WA}
      \resumeItemListStart
        \resumeItem{Developed internal API endpoints for Microsoft Defender XDR customer onboarding, offboarding, and metadata retrieval using Azure Functions}
        \resumeItem{Utilized React with TypeScript to develop a responsive frontend for the Microsoft Defender XDR onboarding wizard and settings page}
        \resumeItem{Rewrote legacy Kockout code in React for Microsoft Sentinel settings and pricing pages}
        \resumeItem{Contributed to the upkeep of services by implementing monitors, writing unit tests, and contributing to internal bug bashes}
      \resumeItemListEnd

      \resumeSubheading
      {Calimetrix}{Jan 2021 -- June 2021}
      {Software Development Intern}{Madison, WI}
      \resumeItemListStart
        \resumeItem{Trained a YOLOv4 object detection model to automate the detection of defects in MRI test objects and reduce time needed for quality control of MRI test objects}
        \resumeItem{Leveraged open source libraries to convert MRI scans to 2D images, align images to a template, and detect defects}
        \resumeItem{Utilized version control and SDLC best practices to create defect detection application with object detection model}
      \resumeItemListEnd
   \resumeSubHeadingListEnd

%----------- RESEARCH PROJECTS-----------
\section{Research Projects}
  \resumeSubHeadingListStart
      \resumeSubheading
      {Informatics Skunkworks Group -- University of Wisconsin}{Jan 2020 -- Dec. 2022}
      % {University of Wisconsin - Madison}{}
      {Pancreatic Cyst Classification - Undergraduate Researcher}{Jan 2020 - Jan 2021}
      \resumeItemListStart
        \resumeItem{Developed machine learning models to classify patients’ pancreatic tumors as mucinous/nonmucinous and malignant/benign to reduce unnecessary surgeries}
            \resumeItem{Created XGBoost and random forest models with oversampling and undersampling using Python and Scikit-learn}
            \resumeItem{Analyzed and processed 496 features for 103 pancreatic tumor patients from University Hospital’s dataset}
            \resumeItem{Acted as a lead of a 3-student development team to delegate tasks and maintain development schedule}
            \resumeItem{Utilized SHAP to analyze feature impact on model output}
      \resumeItemListEnd
      
   \resumeSubSubheading
    {Renal AML Classification and Regression - Undergraduate Researcher}{Jan 2022 - May 2022
}    \resumeItemListStart
        \resumeItem{Guided two subteams of undergraduate students in developing machine learning models}
        \resumeItem{Developed and evaluated machine learning and Neural Network models to predict growth rate of renal AMLs and to classify renal AMLs as high or low growth}
    \resumeItemListEnd

   \resumeSubSubheading
    {Renal Cell Carcinoma Neural Network - Undergraduate Researcher}{Sep 2022 - Dec 2022}
    \resumeItemListStart
        \resumeItem{Developed data transformation pipeline to clean data and make MRI scans usable for Neural Network}
        \resumeItem{Utilized transfer learning to train Convolutional Neural Network to classify Renal Cell Carcinoma as low or high grade}
    \resumeItemListEnd
   \resumeSubHeadingListEnd

%-----------PUBLICATIONS-----------
\section{Publication}
  \resumeItemListStart
      \resumeItem{Awe, A.M., \textbf{Vanden Heuvel, M.M.}, Yuan, T. et al. ``Machine learning principles applied to CT radiomics to predict mucinous pancreatic cysts,'' \textit{Abdominal Radiology}, vol. 47, pp. 221–231, 2022. \href{https://doi.org/10.1007/s00261-021-03289-0}{https://doi.org/10.1007/s00261-021-03289-0}}
  \resumeItemListEnd

%-----------PROGRAMMING SKILLS-----------
\section{Technical Skills}
 \begin{itemize}[leftmargin=0.15in, label={}]
    \small{\item{
     \textbf{Languages}{: Python, JavaScript, C\#, Java, HTML, CSS, MATLAB} \\
     \textbf{Frameworks}{: React, TypeScript, .NET, PyTorch} \\
     \textbf{Developer Tools}{: Git, Azure, VS Code, Visual Studio, Eclipse} \\
     \textbf{Libraries}{: scikit-learn, pandas, NumPy, Matplotlib, SHAP, TensorFlow, Recharts}
    }}
 \end{itemize}


%-------------------------------------------
\end{document}
